\documentclass[12pt]{article}
\usepackage{geometry}
\geometry{a4paper, margin=1in}
\usepackage{enumitem}
\usepackage{hyperref}
\usepackage{tocloft}
\usepackage{graphicx}
\hypersetup{
    colorlinks=true,
    linkcolor=blue,
    urlcolor=cyan,
}

\title{CS3338 Group 3 Design Specification\\\textbf{Want2Remember}}
\author{}
\date{}

\begin{document}

\maketitle
\thispagestyle{empty}
\newpage

\tableofcontents
\newpage

\section{Design Breakdown}

\subsection{Home Page}
\textbf{Purpose:}The home page will provide navigation to other pages and a brief overview of created memories.

\textbf{Features:}
\begin{itemize}
    \item List of all memories created by the user, displayed with details like title and date.
    \item Intuitive interface to ensure ease of use for users.
    \item Links to navigate to Memory Creation, More Detail, and Memory Contact pages.
\end{itemize}

\subsection{Memory Creation Page}
\textbf{Purpose:}Allows users to create and save new memories with detailed information.

\textbf{Features:}
\begin{itemize}
    \item Form for memory input, including fields like title, description, tags, and categories.
    \item Firebase integration to store data in a secure backend.
\end{itemize}

\subsection{More Detail Page}
\textbf{Purpose:} The More Detail Page will show the user all the details of the memory as well as the option to edit the information

\textbf{Features:}
\begin{itemize}
    \item Displays all details of a selected memory.
    \item Edit button to allow modifications to the memory’s title, description, or other fields.
    \item Navigation back to the Home Page or other pages after editing.
\end{itemize}

\subsection{Memory Contact Page}
\textbf{Purpose:}The Memory Contact Page will show the user all contacts they have saved

\textbf{Features:}
\begin{itemize}
     \item Displays all saved contacts relevant to the user’s memories.y.
    \item Option to add, edit, or delete contact details.
    \item Links to a detailed contact view for individual contact information.
\end{itemize}
\section{Development Tools}
\begin{itemize}
\item JavaScript and ReactJS: \newline 
Used for the development of the web application. These frameworks enable component-based development and easy code reuse between the web and mobile applications
    
 \item Github:
 GitHub was used as the centralized repository for storing the source code and project documentation. It ensures versioned and collaborative development, allowing team members to access, contribute, and review code changes. It utilizes branches for parallel development, bug fixes, and feature implementations.
    
 \item JIRA and Agile Development Technology : Utilized for project management, receiving incoming tasks, task monitoring and assignment
    
    \item Firebase:
    Used backend platform, Firestore, to store the user’s data which can integrate with the frontend efficiently, enabling seamless communication between the user interface and database. The integration with Firestore not only facilitates real-time data synchronization but also ensures a responsive and dynamic user experience.

\item TestRail: Used for test runs and track test results
    
\end{itemize}
\section{Future Work}
\begin{itemize}
    \item \textbf{Improve Caregiver Features:} Some of the features that will help caregivers include GPS tracking of supported users and automatic role assignments. Another feature includes allowing the caregiver to remotely control the user’s app to showcase to the care receiver how to perform a specific task.
    \item \textbf{Medication Tracking:} Users will be able to set up reminders for medication and medication administration and save things like proofs of prescriptions.
    \item \textbf{Machine Learning/AI Features:} This can include using AI to improve interactivity with the app or implementing smart reminders that can further support people with cognitive impairments.
\end{itemize}

\end{document}


